\textbfsf{Resumo}\\
Linguagens de programação interpretadas tem trocado performance por
maior expressividade, flexibilidade, dinamicidade, entre outros.
Reescrita de trechos de código críticos em linguagens compiladas tem
sido empregada como meio de reduzir o impacto da máquina virtual no
tempo de execução de programas. Diante disso, um compilador JIT para a
linguagem de programação Tcl é proposto como forma de diminuir a
necessidade de tal reescrita de código. Um modo misto de execução é
escolhido, fazendo com que execução de código de máquina, gerado por
esse compilador dinâmico, e a interpretação pura se
alternem. Procedimentos são definidos como os limites de inicio e
término de compilação, e a compilação baseada em regiões é rapidamente
mencionada. Simplicidade, manutenibilidade e flexibilidade são
características que serão seguidas na construção dessa ferramenta.
Algumas conclusões iniciais a respeito de máquinas virtuais baseadas
em pilha e também sobre aquelas baseadas em registradores são feitas
de acordo com o desenvolvimento do projeto até o momento.

\quad\\
\quad\\
\textit{Palavras-chave}: Tcl, JIT, código de máquina
