
{
\Large
\begin{center}
\textbf{Abstract}
\end{center}
}
Dynamic compilation has been used in implementations of interpreted
programming languages that seek higher performance. In order to bring
this technique to the \texttt{Tcl} language, a non-optimizing JIT
compiler for a subset of it is proposed.
Our choice was for the mixed mode of execution, this way the execution
may alternate between the machine code generated by this dynamic compiler
and the pure interpretation.
Procedures as whole were taken as the compilation unit, where the
compilation of those are fired upon hitting a limit of successful interpreted
executions. Quadruples were chosen to form the intermediate representation
and these were used till the code generation phase for the IA--32
architecture.
Simplicity, maintainability and flexbility were the
characteristics followed in the development of this tool.
To measure the performance of the compiler, six specific benchmarks
were developed. On average, the collected data indicate an improvement
between 1.25 and 23.55 times in relation to the execution time by the
interpreter. The compilation time stayed under 80 micro seconds. Also,
the generated code reduced up to 96.9\% the amount of executed machine
instructions in comparison to the interpreter.

\quad\\
\quad\\
\textit{Keywords}: Tcl, JIT, machine code

\pagebreak
