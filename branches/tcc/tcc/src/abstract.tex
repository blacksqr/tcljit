
{
\Large
\begin{center}
\textbf{Abstract}
\end{center}
}
Interpreted programming languages have been exchanging performance by
greater expressivity, flexibility and dinamicity. The rewrite of
critical code in compiled languages has been deployed as a mean of
reducing the impact of the virtual machine upon the execution time of
programs.

Hence, a non-optimizing JIT compiler for a subset of the \texttt{Tcl}
language is proposed as a mean of trying to decrease the necessity of
this code rewrital.
Our choice was for the mixed mode of execution, this way the execution
may alternate between the machine code generated by this dynamic compiler
and the pure interpretation.
Procedures as whole were taken as the compilation unit, where the
compilation of those are fired upon hitting a limit of successful interpreted
executions. Quadruples were chosen to form the intermediate representation
and these were used till the code generation phase for the IA--32
architecture.
Simplicity, maintainability and flexbility were the
characteristics followed in the development of this tool.

To measure the performance of the compiler, six specific benchmarks
were developed. On average, the collected data indicate an improvement
between 1.25 and 23.55 times in relation to the execution time by the
interpreter.

\quad\\
\quad\\
\textit{Keywords}: Tcl, JIT, machine code

\pagebreak
