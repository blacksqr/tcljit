\chapter{Ambientes de Execução}

% \begin{itemize}
% \item ambientes interpretados
% \item ambientes compilados
% \item ambientes mistos
% \end{itemize}

% Idéias de quando fui deixar cartucho para recarregar (*):
% * Mostrar um diagrama geral de backend e frontend
% * Falar de uma forma geral deles, falar de fases de otimização
% * Falar que parte é comum a todos (analise lexica, analise sintatica e
% analise semantica).

% --> Ou seria na verdade a tabela de símbolos e coisas assim ? Essa
% seria comum a todos (quase)

De modo geral, uma linguagem de programação requer, além da
especificação da própria linguagem, um compilador e uma plataforma de
suporte. Por plataforma de suporte entende-se ambiente de execução,
cuja funcionalidade é possibilitar a execução de programas. Cada
linguagem determina quais serviços são necessários por
parte de ambiente de execução, mas atualmente é comum encontrar os
seguintes serviços:
\begin{itemize}
\item \textbf{Gerenciamento de memória}. Permitir, ao menos, alocar,
  utilizar e liberar memória da pilha e do \textit{heap} de forma
  direta ou indireta.
\item \textbf{Gerenciamento da pilha de execução}. ...
\item \textbf{Gerenciamento de \textit{threads}}. Permitir ...
%\item \textbf{Verificação de tipos em tempo de execução}. ...
\end{itemize}
... Interação desses serviços com o sistema operacional ...


Pode-se distinguir entre três tipos de ambientes de execução: 


%% Já tinha em outro pdf
Em um ambiente interpretado, toda a execução do código fonte se dá em
tempo de execução sem a necessidade de se gerar código de máquina.
Opcionalmente, pode-se transformar o código inicial
em um código intermediário -- bytecode por exemplo -- que pode ser mais
eficientemente interpretado. Esse tipo de ambiente costuma fornecer
alta portabilidade entre diferentes sistemas operacionais e
arquiteturas, código fonte reduzido e agilidade de
desenvolvimento. Todos esses benefícios acabam implicando no declínio
da performance de tais linguagens, sendo assumido a penalidade de uma
ordem de grandeza quando comparado a ambiente compilados
 \cite{jit_eq_betterlate}.

Diferentemente, ambientes compilados utilizam-se de compiladores
estáticos, combinados com assemblers e \emph{linkers} (ligadores), que
realizam a tradução do código fonte para código de máquina específico
para uma arquitetura. Tipicamente (XXX ou sempre ?) esses compiladores
são ditos offline pois tem disponibilidade de todo o código fonte, não
sendo estritamente necessário a tomada de decisões que podem vir a
implicar em resultados não ótimos. Além disso, o compilador aqui
disponibiliza de uma certa folga em relação ao tempo disponível para
se realizar otimizações pois é feito somente a cada recompilação e em
nada acrescenta ao tempo de execução do programa.

Um terceiro tipo de ambiente utiliza os dois mencionados
anteriormente, com a diferença na substituição de compiladores
estáticos por compiladores dinâmicos. Uma linguagem de programação que
faz uso desse ambiente híbrido consegue manter a
portabilidade inicialmente prevista até que seja decidido compilar para código
máquina em tempo de execução parte do código fonte presente em código
intermediário ou não. Os compiladores nesse
tipo de ambiente devem ser relativamente mais inteligentes, não
podendo se dar ao luxo de aplicar todas otimizações, ou mesmo qualquer
otimização, implementadas a qualquer momento pois isso aumentaria em
muito o tempo de execução do programa. Esse ambiente híbrido será o
foco do trabalho aqui descrito.
%% Fim


\section{Compiladores}
\label{sec:compiladores}

\begin{itemize}
\item estrutura
\item compiladores otimizadores
\end{itemize}


\section{Interpretadores}
\label{sec:interp}

\begin{itemize}
\item estrutura do ambiente
\end{itemize}

* Mostrar um diagrama com o ciclo básico de uma máquina virtual:
  - Busca próxima instrução, decodifica e a interpreta


\section{Ambientes Mistos}
\label{sec:hibrido}

\begin{itemize}
\item estrutura
\item compiladores dinâmicos (princípio de execução)
\item questões relacionadas com compilação dinâmica
\begin{itemize}
\item quando/o que/quais otimizações
\end{itemize}
\end{itemize}
