\section{Introdução}

Linguagens de programação interpretadas tem sacrificado performance
em favor de um alto nível de abstração do funcionamento da máquina
envolvida, possibilitando maior expressividade, facilidade de
desenvolvimento, portabilidade, flexibilidade, dinamicidade, entre outros.
Especificamente a linguagem de programação \texttt{Tcl}
(\textit{Tool Command Language}), teve como um de seus maiores
objetivos a facilidade de incorporação (\textit{embedding}) a
programas que desejassem ter uma linguagem de comandos
\cite{ousterhout_89}. Uma interface simples e extensível para
aplicações em linguagem \texttt{C} era o fator motivante de seu uso.
Programas inteiramente em \texttt{Tcl} eram vistos como pequenos
scripts, muitos talvez de uma linha no máximo \cite{ousterhout_89}.
Entretanto, aplicações em \texttt{Tcl} com milhares de linhas,
como por exemplo exmh \cite{exmh}, OpenACS \cite{openacs} ou mesmo a
suíte de testes do GDB (\textit{GNU Debugger}) \cite{gdb_testsuite},
tem surgido e a reescrita de trechos críticos em \texttt{C} vem sendo
aplicada para reduzir o impacto da máquina virtual no tempo de execução.

Outra forma de melhorar o desempenho de linguagens e, portanto,
reduzir a necessidade de reescrita de código em linguagens compiladas,
é através da utilização da compilação JIT (\textit{Just-In-Time}) que
realiza tradução de código sob demanda. No caso desse trabalho, estamos
interessados na tradução de \textit{bytecodes} da máquina virtual
\texttt{Tcl} para código de máquina durante o tempo de execução.
Informações acerca do programa em execução são coletadas
conforme necessário para dirigir a compilação dinâmica. Portanto,
linguagens tipicamente difíceis de serem analisadas e compiladas
estaticamente, devido a uso de, por exemplo, tipagem dinâmica ou
escopo dinâmico, ganham a oportunidade de melhoria de desempenho
com uso de tal sistema de compilação.

O trabalho presente pretende melhorar o desempenho da linguagem
\texttt{Tcl}, reduzindo o tempo de decodificação e interpretação
de \textit{bytecodes} através da implementação e implantação de um
compilador JIT na mesma. Diversos trabalhos, como
\cite{deutsch84efficient} para a linguagem \texttt{Smalltalk},
Jalapeño \cite{jalapeno_1} para \texttt{Java}, Psyco \cite{psyco}
para \texttt{Python} ou a implementação do SELF-93 \cite{holzle}, já
demonstraram resultados bastante significativos ao implantar tal
método de compilação. Pretende-se ainda manter um nível de
manutenibilidade e flexibilidade adequado, permitindo extensões e
futuro desenvolvimento sobre esse trabalho inicial.

As seções seguintes dessa proposta estabelecem o que, e como, será
feito no decorrer do trabalho.

\section{Objetivos}

Construir um compilador JIT que gere código de máquina para a
arquitetura IA-32 durante a interpretação de programas \texttt{Tcl}
é o grande objetivo desse trabalho.

Certas construções de linguagem existentes na \texttt{Tcl} dificultam
o processo de compilação (estática ou dinâmica) e já foram alvo de
discussão em trabalhos passados \cite{sah_blow} \cite{sah_tc}, levando
até mesmo a decisão de se criar uma outra linguagem similar \cite{rush_lang}
eliminando as construções mais problemáticas. Por essa razão, o
trabalho discutido aqui pretende focar em apenas um subconjunto
significativo da linguagem.

O projeto também pretende explorar como sistemas compactos, simples e com
menos recursos se comparados a frameworks como LLVM (
\textit{Low Level Virtual Machine}) \cite{llvm_2002}, influenciam no desempenho
geral. Porém, apesar da simplicidade em vista, espera-se que o
trabalho desenvolvido possa servir de base para expansão e criação de
novos recursos.


\section{Proposta}

Para chegar ao código de máquina final, esse projeto pretende em
primeiro momento estudar e definir qual subconjunto da linguagem pode
se beneficiar mais com tal técnica. O subsistema de Entrada/Saída, por
exemplo, dificilmente ganharia em desempenho com compilação dinâmica
uma vez que o tempo gasto para transferência e aguardo por dispositivos
costuma ser muito maior que o tempo das outras operações envolvidas.
Por outro lado, operações aritméticas podem ser bastante favorecidas.
O trabalho feito no Psyco, mostra melhoria
de 109 vezes no tempo de execução para aritmética de inteiros e 10,9
vezes em aritmética de ponto flutuante \cite{psyco}.

Em paralelo ao requisito acima, será iniciado o projeto e
implementação de uma representação intermediária de baixo
nível. Queremos utilizar um conjunto reduzido de instruções, assim
como é feito na GNU lightning \cite{gnu_lightning}, para construir uma
árvore de instruções mais próximas da máquina alvo. O uso de
árvore aqui é devido a existência de diversos métodos para seleção de
instruções que trabalham sobre essa forma de representação
\cite{ir_tree_parsing}. O sistema de compilação dinâmica precisa ser
capaz de determinar quando iniciar e parar a construção de tal árvore.
Nesse projeto a intenção é fazer uso dos
limites de um procedimento como pontos de inicio e parada. Ainda aqui,
é importante que o sistema colete informações, como de tipos
utilizados, necessárias de forma a simplificar o código gerado, para
que não repliquemos os resultados do trabalho feito em
\cite{vitale_catenation}.

Após as etapas acima, chegamos na fase de seleção de instruções
seguida de alocação de registradores. Há diversos algoritmos, como
\textit{Maximal Munch}, seleção com uso de programação dinâmica ou
NOLTIS \cite{noltis}  para seleção de instruções além de vários
outros, por coloração de grafos ou mesmo através de uma varredura
linear \cite{linear_scan_regalloc}, para alocação de registradores.
Ainda não está decidido quais dos algoritmos serão aplicados nesse
projeto, porém as metas são duas: que a implementação permita utilizar
diferentes algoritmos e que o tempo gasto para execução deles
seja compatível com um sistema que consome tempo de execução do programa.

Finalmente temos a geração de código. O foco desse trabalho é a
arquitetura IA-32, que possui uma ampla quantidade de instruções, extensões e
diversas formas de endereçamento. % \cite{intel_basicarch}.
Porém, não pretendemos fazer uso de todos os recursos disponíveis de
forma a tornar factível a criação do gerador.
Idéias e trechos de trabalhos anteriores, como Harpy \cite{harpy}
ou o \textit{Tiny Code Generator} (incluído nas versões mais recentes do
QEMU \cite{qemu}), podem ser reutilizadas aqui.
Ainda, de forma semelhante com a etapa anterior, a infraestrutura
permitirá a implantação de geração de código para outras arquiteturas.


\section{Revisão Bibliográfica}

A linguagem \texttt{Tcl} já obteve ganhos de performance em diferentes
estudos feitos. Um deles, que atualmente faz parte da implementação da
linguagem, descrito em \cite{tcl_bytecode}, é a geração e a
interpretação de \textit{bytecodes}. Anterior a esse trabalho, foi
demonstrado em \cite{sah_tc} que o \textit{parsing} do código
realizado a todo momento para sua reinterpretação e também a conversão
excessiva entre tipos, já que a \texttt{Tcl} trata tudo como string,
eram os grandes consumidores do tempo de execução. Com esse trabalho
feito, a linguagem passou a utilizar representações duplas para os
valores presentes na execução do programa. Uma representação é
interna, possivelmente mais eficiente para se trabalhar. A outra é a
típica representação em string que a linguagem sempre usou. Caso uma
delas não esteja disponível, a outra é utilizada para recriar essa
representação se necessário.

Um trabalho mais recente, descrito em \cite{vitale_catenation}, lida
com a eliminação do \textit{overhead} de decodificação dos
\textit{bytecodes}, introduzido pelo trabalho descrito anteriormente,
fazendo uso de \textit{templates} que contém as instruções em
código nativo utilizadas para interpretar cada \textit{bytecode}.
Esse código é obtido através da compilação do próprio interpretador
\texttt{Tcl} e cada \textit{template} é copiado múltiplas vezes,
numa área de memória alocada em tempo de execução, conforme a quantidade de
cada \textit{bytecode} gerado. Nesse trabalho o interpretador foi
modificado de forma a sempre executar somente tal código formado por
uma concatenação de \textit{templates}, eliminando o \textit{overhead} de
decodificação. Demonstrou-se que em certos testes o desempenho da
linguagem pode melhorar em até 60\% com a aplicação dessa técnica.
Esse trabalho é provavelmente o mais próximo, quando considerando
somente a \texttt{Tcl}, do que se pretende produzir aqui.
Ele não gera código, mas cópia código já gerado por um compilador
estático e replica conforme necessário, fazendo os devidos
ajustes, em tempo de execução. Por um lado o tempo de ``compilação'' é
bastante baixo, porém, não dá espaço para técnicas de otimização e assim
limita o potencial de melhoria de desempenho.

Outros trabalhos, para diferentes linguagens, se assemelham mais com a
proposta aqui discutida. A busca por máquinas virtuais de alta
performance tem, atualmente, se dirigido principalmente a linguagem
\texttt{Java}. É comum a presença de compiladores JIT nessas máquinas
virtuais, cada um com diferentes características. A JUDO \cite{judo}, faz uso
de compilação dinâmica com dois tipos de compiladores e coleta
informações em tempo de execução. O primeiro desses compiladores é um
mais simples, que gera código rapidamente, destinado a compilação
de métodos invocados pela primeira vez. O segundo compilador é
utilizado quando informações coletadas indicam que certos métodos
são executados muito frequentemente e, portanto, estes podem se
beneficiar com a aplicação de otimizações. Essa recompilação dinâmica
é feita com o intuito de balancear o tempo gasto na compilação com o tempo
efetivamente gasto na execução do programa. Esse sistema trabalha com
a compilação de métodos por inteiro, assim como o trabalho proposto
aqui. Já o trabalho discutido em \cite{suganuma_pldi_2003} avalia a
aplicação de compilação dinâmica a regiões de código, evitando
a compilação de trechos raramente executados. Além disso, esse sistema
utiliza um modo misto de execução, onde interpretação e execução de
código nativo se alternam. Nesse ponto, nosso trabalho e aquele em
\cite{suganuma_pldi_2003} se assemelham.

% LLVM talvez


\section{Metodologia}

A metodologia será experimental, onde será implementado uma
``ferramenta'' e então avaliada para provar a tese. A seguir são
resumidas as etapas previstas para a construção desse trabalho.
\quad\\

\textbfsf{Levantamento bibliográfico}

Selecionar artigos, livros e manuais a serem utilizados ao longo da
escrita textual e também na escrita de código. Essa etapa ocorre em
paralelo a todas as outras descritas adiante, pois novos
documentos serão selecionados conforme o trabalho avança em direção
aos objetivos estabelecidos.

\textbfsf{Especificação do compilador JIT}

Estabelecer um formato para a representação intermediária, avaliar e
definir algoritmos para seleção de instruções e alocação de
registradores serão realizados aqui. Essa etapa ocorrerá logo no inicio
do projeto e também durante o progresso da implementação do compilador
de forma a validar as escolhas feitas.

Ainda aqui será estabelecido como a implementação do compilador deverá
acontecer, mantendo o objetivo de permitir que algoritmos
alternativos venham a ser utilizados no lugar daqueles aqui
selecionados e possibilitando que diferentes arquiteturas explorem os
recursos disponíveis de maneira adequada.

\textbfsf{Especificação da arquitetura da \texttt{Tcl}}

Inicialmente será feito uma análise para verificar qual subconjunto da
linguagem cabe ao trabalho proposto. A máquina virtual deve então ser
ajustada para conseguir determinar quando um procedimento se enquadra no
subconjunto definido ou não, decidindo entre inserir o
procedimento como candidato a compilação JIT ou não. Outro ajuste está
relacionado com a notificação ao compilador JIT sobre comandos que são
redefinidos, possivelmente alterando o código anterior associado a ele, e
portanto precisam ser recompilados. A máquina virtual da \texttt{Tcl}
já conta com um contador para cada comando implementado, do usuário ou
não, que é incrementado a cada redefinição e, assim, esse
compilador fará uso desse recurso.

Essa etapa será feita em diversos momentos, conforme a implementação
do compilador progride, validando o que foi já definido.

\textbfsf{Implementação do Compilador}

Essa etapa refere-se a implementação da representação intermediária de
baixo nível, seleção de instruções, alocação de registradores e
geração de código de máquina para a arquitetura IA-32. Os algoritmos
utilizados para seleção de instruções e alocação de registradores
serão aqueles definidos na etapa de especificação do compilador JIT.

\textbfsf{Acoplamento do compilador ao ambiente de execução}

% XXX

Definir como detectar e lidar com exceções no código de máquina gerado,
possibilitando a retomada da execução por meio de interpretação pura.

XXX

\textbfsf{Avaliação}

Pretende-se fazer uso da suíte de \textit{benchmarks}
tclbench \cite{tcllib_bench} já que essa tem sido utilizada para verificar
e comparar a performance da \texttt{Tcl} ao longo de vários
anos. Outras formas de avaliação, como tamanho de código gerado e
tempo gasto para compilação por \textit{bytecode} também
poderão ser aplicadas conforme o tempo permitir. Planeja-se iniciar
essa etapa após a implementação, pois há dependência de um sistema JIT
em funcionamento para se executar a avaliação.

\textbfsf{Elaboração da Monografia}

Partes textuais da monografia, que não dependam de resultados obtidos
através da realização das etapas de implementação, serão escritas quão
antes possível. Experimentos realizados ao longo do desenvolvimento
influenciarão diversas seções do trabalho escrito e, portanto, o
desenvolvimento dessa parte se dará ao longo de todo o tempo disposto
ao trabalho.


\section{Cronograma de Execução}

\begin{tabular}{m{55mm}|m{58mm}}
%\begin{tabular}{l|l}
\textbfsf{Etapa} & Mai\quad Jun\quad Jul\quad Ago\quad Set\quad Out\\
\hline
Levantamento bibliográfico & \rule{51 mm}{3mm}\\
\hline
Especificação do compilador JIT & \rule{4mm}{3mm}\hspace{11mm}\rule{3mm}{3mm}\hspace{11mm}\rule{3mm}{3mm} \\
\hline
Especificação da arquitetura Tcl & \rule{4mm}{3mm}\hspace{14mm}\rule{4mm}{3mm}\hspace{10mm}\rule{4mm}{3mm}\\
\hline
Implementação do Compilador & \hspace{4mm}\rule{40mm}{3mm}\\
\hline
Acoplamento do compilador ao ambiente de execução & \hspace{44mm}\rule{10mm}{3mm}\\
\hline
Avaliação & \hspace{44mm}\rule{7mm}{3mm}\\
\hline
Escrita da Monografia & \rule{57mm}{3mm}
%\hline
%Otimizações* & XXX\\
\end{tabular}


% Referências
% Hack: altera \section* para \section temporariamente
\let\stdsection\section
\def\section*#1{\stdsection{#1}}
%\bibliographystyle{sbc}
\bibliography{biblio}
\let\section\stdsection


%\section{Bibliografia Sugerida}
% Para possivelmente ser investigada no decorrer do trabalho
%\begin{enumerate}
%\item \textbfsf{Artigos}
%  \begin{itemize}
%    \item One-Pass, Optimal Tree Parsing - With or Without Trees
%    \item Division by Invariant Integers using Multiplication
%    \item Linear Scan Register Allocation
%    \item Tailoring Graph-coloring Register Allocation For Runtime
%      Compilation
%    \item Dynamic Native Optimization of Interpreters
%    \item Optimally Profiling and Tracing Programs
%    \item NOLTIS
%    \item Near-optimal instruction selection on dag
%    \item Fast and Flexible Instruction Selection with On-Demand Tree-Parsing Automata
%  \end{itemize}
%\item \textbfsf{Livros}
%  \begin{itemize}
%    \item The Compiler Design Handbook - 2nd (capítulos 16, 17, 19,
%      21)
%    \item Compilers: Principles, Techniques and Tools - 2nd (capítulos
%      8, 9)
%    \item Enigeering a Compiler (capítulos 9, 10, 11)
%    \item Modern Compiler Implementation in C (capítulos 8, 9, 10, 11)
%    \item Advanced compiler design and implementation (capítulos 4, 6)
%  \end{itemize}
%\item \textbfsf{Manuais}
%  \begin{itemize}
%    \item Intel 64 and IA-32 Architectures Software Developer's Manual
%      - Volume 2A: Instruction Set Reference (A-M)
%    \item Intel 64 and IA-32 Architectures Software Developer's Manual
%      - Volume 2B: Instruction Set Reference (N-Z)
%  \end{itemize}
%\end{enumerate}
