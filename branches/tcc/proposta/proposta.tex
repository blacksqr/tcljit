\documentclass[12pt]{article}
\usepackage[brazil]{babel}
\usepackage[utf8]{inputenc}
\usepackage{times}
\usepackage{txfonts}
\usepackage[a4paper,top=3cm,left=3cm,right=3cm,bottom=3cm]{geometry}
\usepackage{setspace}
\usepackage[parfill]{parskip}
\usepackage{titlesec}
\usepackage{sectsty}

\setlength{\parindent}{0pt}
\setlength{\parskip}{6pt}
\onehalfspacing
\titlespacing{\section}{0pt}{6pt}{6pt}
\makeatletter
\sectionfont{\fontfamily{phv}\fontseries{b}\fontsize{14}{\f@baselineskip}\selectfont}
\subsectionfont{\fontfamily{phv}\fontseries{b}\fontsize{13}{\f@baselineskip}\selectfont}

\newcommand{\instituicao}{Universidade Estadual de Maringá \\
Centro de Tecnologia \\
Departamento de Informática \\
Bacharelado em Ciência da Computação \\
\hfill}
\newcommand{\titulo}{
Trabalho de Conclusão de Curso - TCC 2010\\ [5cm]
Tcl JIT}
\newcommand{\areaconcentracao}{Compiladores}
\newcommand{\autor}{Guilherme Henrique Polo Gonçalves}
\newcommand{\orientador}{Prof. Dr. Anderson Faustino da Silva}
\newcommand{\local}{Maringá}

\newcommand*{\textbfsf}[1]{\textbf{\textsf{#1}}}

\begin{document}

\begin{titlepage}
  \vfill
  \begin{center}
  {\Large \textbfsf \instituicao} \\
  {\large \textbfsf \titulo}\\[5cm]
  Área de Concentração: \areaconcentracao \\
  Aluno: \autor\\ % XXX
  Orientador: \orientador\\ % XXX
  \vfill
  \end{center}
\end{titlepage}


\section{Objetivos gerais}
Melhorar o desempenho de programas escritos na linguagem de
programação \texttt{Tcl}
(\emph{Tool Command Language}) através da criação e implantação de um
compilador JIT (\textit{Just-In-Time}) na mesma.


\section{Justificativas}
Diversos trabalhos já demonstraram a possibilidade de aumentar o
desempenho de linguagens com a aplicação de compilação
dinâmica. Alguns desses projetos serão mencionados, e acredita-se que a
\texttt{Tcl} possa também gerar resultados relevantes.

Uma implementação da linguagem \texttt{Smalltalk-80} descrita em
\cite{deutsch84efficient} traduz dinamicamente bytecodes da máquina
virtual da \texttt{Smalltalk-80} para código nativo, obtendo,
dependendo da estratégia utilizada, entre 1.45x e quase 2.0x
melhor tempo de execução quando comparado a utilização de apenas
interpretação.

Um outro projeto, o Psyco \cite{psyco}, demonstra resultados
excelentes em casos específicos de aritmética de inteiros, reduzindo
em 109x o tempo de execução, e também em aritmética de
ponto flutuante, obtendo melhoria de 10.9x no tempo gasto, para a linguagem
 \texttt{Python}.

Já a linguagem \texttt{Java} tem recebido considerável atenção
em relação ao tópico discutido e várias máquinas virtuais foram
criadas com a intenção de melhorar seu desempenho. Um exemplo é a Jalapeño JVM
(\textit{Java Virtual Machine}), onde o trabalho descrito em
\cite{jalapeno_1} apresenta uma comparação entre seu tempo de execução
utilizando um compilador otimizador dinâmico contra
interpretação da JVM IBM Developer Kit. Nessa comparação, a Jalapeño
demonstra vantagens de 7 a 29x no tempo de execução.


\section{Breve fundamentação}
Compilação JIT se refere a tradução de código conforme
necessário. No caso desse trabalho, estamos interessados na tradução de
\textit{bytecodes} da máquina virtual \texttt{Tcl} para código de máquina
durante o tempo de execução.

Algumas linguagens, como \texttt{SELF} ou \texttt{Tcl}, são difíceis de serem
analisadas e compiladas estaticamente de forma eficiente. Uma das
causas pode ser a utilização de tipagem dinâmica. O resultado é
que tais linguagens não conseguem alcançar boa performance devido, em partes, ao
\textit{overhead} causado pela interpretação. São nessas linguagens de
programação onde um compilador JIT se mostra mais necessário, pois a
utilização de informações coletadas durante a execução do programa
torna possível a geração de código mais eficiente.

Um sistema que aplica compilação dinâmica funciona, basicamente,
de uma das formas seguintes. No caso do \texttt{SELF-93} \cite{holzle}, os
\textit{bytecodes} são convertidos para código de máquina antes de se
executar qualquer método e a interpretação nunca é utilizada. Essa
tradução é feita de forma rápida, fazendo com que o tempo de execução
total não seja muito prejudicado, porém o código não é tão
eficiente. Entretanto, métodos que executarem muito frequentemente serão
notificados para o sistema de recompilação que, então, determina
se o compilador otimizador deve ser utilizado ou não. Uma outra forma
é a utilização de MMI (\textit{Mixed Mode Interpretation}), onde
interpretação e execução de código nativo se alternam. O \texttt{YAPc}
\cite{yapc} funciona dessa forma, o que, no caso do \texttt{Prolog},
significa que apenas as cláusulas utilizadas mais frequentemente serão
compiladas.

Independente da forma de funcionamento implementada, um sistema JIT
elimina parte do \textit{overhead} da interpretação para que o
programa obtenha melhor desempenho. Tempo de decodificação e interpretação de
\textit{bytecodes} podem ser significantemente reduzidos, enquanto que
tarefas relacionadas a, por exemplo, sincronização ou alocação de
memória não são diretamente afetados pelo JIT.

\bibliographystyle{sbc}
\bibliography{biblio}

%\clearpage
\vspace{5cm}

\begin{center}
\vbox{
\clearpage
\thispagestyle{empty}
\null
\begin{flushleft}
\rule{0.55\textwidth}{0.5pt} \par
\autor
\vskip 0.35in
\rule{0.55\textwidth}{0.5pt} \par
\orientador
\vskip 2.5cm
\end{flushleft}
}

\local, \@date
\end{center}

\end{document}
