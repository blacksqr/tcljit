\documentclass[12pt]{article}
\usepackage[brazil]{babel}
\usepackage[utf8]{inputenc}
\usepackage{times}
\usepackage{txfonts}
\usepackage[a4paper,top=3cm,left=3cm,right=3cm,bottom=3cm]{geometry}
\usepackage{setspace}
\usepackage[parfill]{parskip}
\usepackage{titlesec}
\usepackage{sectsty}

\setlength{\parindent}{0pt}
\setlength{\parskip}{6pt}
\onehalfspacing
\titlespacing{\section}{0pt}{6pt}{6pt}
\makeatletter
\sectionfont{\fontfamily{phv}\fontseries{b}\fontsize{14}{\f@baselineskip}\selectfont}
\subsectionfont{\fontfamily{phv}\fontseries{b}\fontsize{13}{\f@baselineskip}\selectfont}

\newcommand{\instituicao}{Universidade Estadual de Maringá \\
Centro de Tecnologia \\
Departamento de Informática \\
Bacharelado em Ciência da Computação \\
\hfill}
\newcommand{\titulo}{
Trabalho de Conclusão de Curso - TCC 2010\\ [5cm]
Tcl JIT}
\newcommand{\areaconcentracao}{Compiladores}
\newcommand{\autor}{Guilherme Henrique Polo Gonçalves}
\newcommand{\orientador}{Prof. Dr. Anderson Faustino da Silva}
\newcommand{\local}{Maringá}

\newcommand*{\textbfsf}[1]{\textbf{\textsf{#1}}}


\begin{document}

\begin{titlepage}
  \vfill
  \begin{center}
  {\Large \textbfsf \instituicao} \\
  {\large \textbfsf \titulo}\\[5cm]
  Área de Concentração: \areaconcentracao \\
  Aluno: \autor\\ % XXX
  Orientador: \orientador\\ % XXX
  \vfill
  \end{center}
\end{titlepage}


\section{Introdução}

A linguagem de programação Tcl (\emph{Tool Command Language}) foi
criada com o propósito de facilitar sua incorporação
(\emph{embedding}) a programas que desejassem ter uma linguagem de
comandos \cite{ousterhout_89}. Uma
interface simples e extensível para aplicações em linguagem C era o
fator motivante de seu uso. Programas inteiramente em Tcl eram
vistos como pequenos scripts, muitos talvez de uma linha no
máximo \cite{ousterhout_89}. Entretanto, a linguagem tem evoluído ao
longo de mais de 20 anos e, com isso, seu uso se expandiu e programas
muito maiores que uma linha como, por exemplo, exmh \cite{exmh},
Tkabber ou Coccinella tem sido criados.

Como em várias linguagens interpretadas, o desempenho da Tcl
pode ser um obstáculo na construção de
programas que exigem bastante do processador (\emph{CPU bound}).
No caso específico de Tcl, esse
tipo de problema tem sido circundado com a reescrita de
trechos críticos na linguagem C e fazendo uso da API
(\emph{Application Programming Interface}) fornecida pela linguagem
para que tais partes possam ser utilizadas pelas outras partes de programas
escritos puramente em Tcl.
Por outro lado, a linguagem Java, um dos exemplos mais conhecidos, optou pela
utilização da compilação dinâmica e usualmente todo código de um
programa Java é escrito puramente em Java.

A compilação dinâmica, ou JIT (\emph{Just-In-Time}), é assim chamada
pois ocorre durante a execução de um programa. Tal programa pode ser
um interpretador para uma linguagem de programação qualquer. Sendo
assim, ao tempo de execução do programa é somado o tempo de todo o
processo de compilação JIT. Com isso fica claro que esse compilador
deve conseguir, eficientemente, identificar o que e quando compilar de
forma que o código de máquina gerado execute em um tempo
significantemente menor. Trabalhos relacionados, como JUDO
\cite{judo}, Psyco \cite{psyco} e YAPc \cite{yapc} relatam
ganhos significantes em tempo de execução para as linguagens Java,
Python e Prolog respectivamente. Espera-se, então, que o trabalho
aqui discutido possa ajudar a melhorar a performance da Tcl cuja
existência da aplicação de compilação JIT para a mesma é desconhecida.

%A compilação dinâmica, ou JIT (\emph{Just-In-Time}), com geração de
%código de máquina em linguagens de programação iniciou-se com a
%modificação do Smalltalk-80 %\cite{bluebook}
%de forma a obter um sistema com
%performance aceitável\cite{deutsch84efficient}.
% Muitos outros
%projetos de propósito similar tem sido construídos desde então.
% Nessa implementação do Smalltalk, um
%procedimento é traduzido de .. v-code .. para código nativo, denominado de
%\emph{n-code}, antes de sua primeira execução \cite{deutsch84efficient}.


\section{Objetivos e Justificativas}

O trabalho proposto tem como objetivo final melhorar o desempenho
de programas escritos em Tcl, sem que o código
precise ser reescrito em uma linguagem como C que, consequentemente,
resulta em perca das
vantagens fornecidas pela linguagem de mais alto nível.

Para alcançar
essa meta, antes um estudo da linguagem Tcl deve ser realizado para que
seja identificado os pontos onde há uma maior penalização causada pela
interpretação. Tendo feito isso será possível focar apenas num
subconjunto da Tcl. A compilação dinâmica irá, então, tratar desses
pontos fazendo com que funções que sejam executadas muito
frequentemente e que também satisfaçam critérios relacionados a
efeitos colaterais causados por comandos da própria linguagem sejam
compiladas para código nativo. Sendo assim, a infra-estrutura desse
compilador JIT deve ser capaz de selecionar adequadamente
procedimentos candidatos a compilação.

Durante a execução do programa deve ser feito a coleta de informações
sobre os parâmetros passados a um procedimento, também será necessário
incluir contadores nos objetos internos que representam procedimentos
para possibilitar a identificação da frequência de execução dos mesmos, e
também é necessário analisar dinâmicamente o código de forma que seja possível
decidir entre compilar um certo procedimento ou não.
A contribuição mais significativa desse trabalho será essa análise
dinâmica de código Tcl, que possibilita a geração eficiente de código
nativo.


\bibliographystyle{sbc}
\bibliography{biblio}

\begin{center}
\vspace{5cm}
\vbox{
\clearpage
\thispagestyle{empty}
\null
\begin{flushleft}
\rule{0.55\textwidth}{0.5pt} \par
\autor
\vskip 0.35in
\rule{0.55\textwidth}{0.5pt} \par
\orientador
\vskip 3cm
\end{flushleft}
}

\local, 31 de Março de 2010
\end{center}

\end{document}
